% Document class and basic setup
\documentclass[12pt]{exam}
% \usepackage[utf8]{inputenc}
% \usepackage[T1]{fontenc}

% Page layout and geometry
\usepackage[margin=0.5in, includefoot, includehead]{geometry}
\usepackage{microtype}
\usepackage{setspace}
\usepackage{tcolorbox}
\tcbuselibrary{breakable, minted}

% Font packages - XeLaTeX/LuaLaTeX vs pdfLaTeX compatibility
\usepackage{ifxetex,ifluatex}
\ifxetex
\usepackage{fontspec}
\setmainfont{Linux Libertine O}[
  Ligatures=TeX,
  Numbers=OldStyle
]
\setsansfont{Linux Biolinum O}[
  Ligatures=TeX,
  Numbers=OldStyle
]
% \setmonofont{Latin Modern Mono}[Scale=MatchLowercase]
\setmonofont{DejaVu Sans Mono}[Scale=MatchLowercase]
\else\ifluatex
\usepackage{fontspec}
\setmainfont{Linux Libertine O}[
  Ligatures=TeX,
  Numbers=OldStyle
]
\setsansfont{Linux Biolinum O}[
  Ligatures=TeX,
  Numbers=OldStyle
]
\setmonofont{Latin Modern Mono}[Scale=MatchLowercase]
\else
\usepackage{libertine}
\fi\fi

% Math packages
\usepackage{amsmath,amssymb,amsthm}
\usepackage{mathtools}
\usepackage{bm}
\usepackage{accents}
\usepackage{miller}

% List and enumeration
\usepackage[shortlabels]{enumitem}
\usepackage{multicol}

% Graphics and figures
\usepackage{graphicx}
\usepackage{tikz}
\usepackage{pgfplots}
\pgfplotsset{compat=1.18}
\usetikzlibrary{shapes}
\usepackage{pgfplots}
\pgfplotsset{compat=1.18}
% \usepackage{svg}
\usepackage{graphicx}
\usepackage{float}
\usepackage{wrapfig}
\usepackage{tikz-3dplot}

% Tables and arrays
\usepackage{booktabs}
\usepackage{multirow}
\usepackage{nicematrix}

% Color and highlighting
\usepackage{xcolor}
\definecolor{answerboxcolor}{RGB}{255,245,245} % Very light red background
\usepackage{empheq}

% Captions and references
\usepackage[hypcap=false,font=small,labelfont=bf,tableposition=top]{caption}
\usepackage{subcaption}
\usepackage[colorlinks=true, allcolors=black]{hyperref}
\usepackage{cleveref}
\usepackage[style=numeric,backend=biber]{biblatex}
\addbibresource{src/refs.bib}

% Units and chemistry
\usepackage{siunitx}
\sisetup{
  group-digits=integer,
  group-minimum-digits=3,
  group-separator={,}
}
\DeclareSIUnit\angstrom{\text{\AA}} % \angstrom is depracted, so define it here.
\usepackage[version=4]{mhchem}
\usepackage{chemformula}

% Code listings
\usepackage{minted}
\usepackage{listings}
\AtBeginEnvironment{minted}{
\fontsize{8}{10}\selectfont}

% Utility packages
\usepackage{cancel}
\usepackage{etoolbox}
\usepackage{pdfpages}
\usepackage{blindtext}
\usepackage{lipsum}

% Custom commands
\newcommand{\fahrenheit}{^\circ{F}}
\newcommand*\widefbox[1]{\fbox{\hspace{2em}#1\hspace{2em}}}
\newcommand{\msout}[1]{\text{\sout{$#1$}}}

% SI Units
\DeclareSIUnit\year{y}

% Professional boxed answer environment
% Creates uniform, centered answer boxes with light red background
\newcommand{\boxedanswer}[1]{
  \begin{center}
    \fcolorbox{black}{answerboxcolor}{%
      \begin{minipage}{0.85\textwidth}
        \vspace{0.5em}
        #1
        \vspace{0.5em}
      \end{minipage}%
    }
  \end{center}
  \vspace{0.5em}
}

% Alternative boxed answer for nested environments (like enumerate)
\newcommand{\boxedanswersmall}[1]{
  \begin{center}
    \fcolorbox{black}{answerboxcolor}{%
      \begin{minipage}{0.75\textwidth}
        \vspace{0.3em}
        #1
        \vspace{0.3em}
      \end{minipage}%
    }
  \end{center}
  \vspace{0.3em}
}

% Additional SI unit for Fahrenheit
\DeclareSIUnit\fahrenheit{\degree F}

\begin{document}

% Include title page
% Title page with no headers/footers
\thispagestyle{empty}

\begin{titlepage}
  \begin{center}
    \vspace*{1cm}

    % Main title
    \Huge
    \textbf{A Review of Density Functional Theory}

    \vspace{0.8cm}

    % Course information
    \LARGE
    MSEN 640-600

    \vspace{2cm}

    % Author information
    \Large
    \textbf{Prepared by:}\\[0.5cm]
    \huge
    \textbf{Nathaniel Thomas} \par
    \Large
    \textit{Texas A\&M University, Department of Nuclear Engineering}

    \vspace{1cm}

    \textbf{Prepared for:}\\[0.5cm]
    \large
    Dr. Xiaofeng Qian

    \vfill

    % University logo
    \includegraphics[width=0.4\textwidth]{"./assets/a&m_logo.pdf"}

    \vspace{1cm}

    % University and date information
    \Large
    Materials Science \& Engineering\\
    Texas A\&M University\\
    \vspace{0.5cm}
    \large
    December 1\textsuperscript{st}, 2025

  \end{center}
\end{titlepage}

% Header and footer setup using exam class commands
% The exam class provides its own header/footer system
\lhead{\textbf{MSEN 640-600}}
\chead{\textbf{A Review of Density Functional Theory}}
\rhead{\textbf{Nathaniel Thomas}}
\lfoot{}
\cfoot{\thepage}
\rfoot{}

% Add header rule
\headrule


\pagebreak

\begin{center}
  \textbf{Abstract}
\end{center}

\begin{spacing}{1.15}
  Density functional theory (DFT) is a computational quantum
  mechanical method that enables
  the simulation of quantum systems involving multiple atoms. It
  provides efficient
  approximations to the solutions of the Schr\"odinger equation.
  Direct numerical solutions
  of the Schr\"odinger equation are typically feasible only for the
  simplest systems; for
  larger systems, such as an iron atom with 26 electrons, exact solutions become
  computationally prohibitive due to the need to track the state of
  every electron.
  DFT addresses this challenge by replacing the many-electron
  wavefunction with the
  electron density, allowing simulations of many-body quantum
  mechanical systems.
  These include not only single atoms with many electrons but also
  multi-atom systems,
  with some simulations involving thousands of atoms and enabling the
  calculation of
  macroscopic material properties using ab-initio techniques.
  However, DFT introduces
  certain limitations, primarily due to the unknown exact form of the
  exchange-correlation
  functional, which must be approximated to describe many-body
  electron interactions.
\end{spacing}

\vspace{0.5cm}
\noindent\textbf{Keywords:} Density functional theory, Quantum
mechanics, Computational materials science, Ab initio quantum chemistry

\pagebreak

\section{Introduction}

% Much of this introductory content is taken from \cite{sholl2009ch1}. \par

Density functional theory is an approximation method for solving the
Schr\"odinger equation. The method builds on the original equation, and
applies several approximations to bring the computational cost of evaluation
down from exponential to cubic growth, at minimal loss in accuracy.
To understand
the significance of this achievement, we must first walk through the origins of
the Schrodinger equation, and show how DFT allows it to be solved
computationally
through several key approximations. \par

The Sch\"odinger equation is one of the most important equations in
modern physics.
It is one of the most important foundational equations in quantum mechanics,
enabling the precise calculation of phenomena in quantum mechanics. Solving
it enables one to precisely predict behavior on an atomic scale. The
Schr\"odinger
equation was first formulated in 1925, and published in 1926
\cite{schrodinger1}. In a series of four publications, Erwin Schr\"odinger was
investigating the structure of the hydrogen atom, and the equation appeared as
follows:

\begin{equation}
  \Delta \psi + \frac{2m}{K^2}\left( E + \frac{e^2}{r} \right)\psi = 0
\end{equation}

Where $\Delta \psi$ is the Laplacian of the wavefunction $\psi$, $m$ is mass,
$K = \frac{h}{2\pi} = \hbar$, $E$ is energy, $e$ is elementary charge, and $r$
is radial distance. We can generalize the Coulomb potential as any potential
function which may be position and time dependent, $\frac{e^2}{r} =
-V(\vec r, t)$,
and substitute in $\Delta = \nabla^2$,
as well as $\hbar$ to recover the modern form of the time-independent
Schr\"odinger equation. With rearrangement:

\begin{equation}
  \left(-\frac{\hbar^2}{2m}\nabla^2 + V(\vec r, t)\right) \psi = E\psi
\end{equation}

We can also introduce the Hamiltonian operator $\hat H =
\left(-\frac{\hbar^2}{2m}\nabla^2 + V(\vec r, t)\right)$, which gives us:

\begin{equation}
  \hat H \psi = E\psi
  \label{eq:tise}
\end{equation}

which is the time-independent Schr\"odinger equation. In a follow up to the
original series, Schr\"odinger formulated the time-dependent Schr\"odinger
equation, recognizing a need to evaluate the time-evolution of quantum systems
to explain the absorption/emission spectra of hydrogen
\cite{schrodinger2}. It takes on the
daunting form:

\begin{equation}
  \Delta _p^{\frac{1}{3}} \sum_l \frac{\delta}{\delta q_l}
  \left(\Delta_p^{-\frac{1}{2}}\sum_k\alpha_{lk} \delta\psi/\delta q_k\right)
  - 8\pi^2V\psi/h^2 \mp (4\pi i/ h)\delta \psi / \delta t = 0
\end{equation}

The first term, while daunting, can be thought of as the Laplacian of the
wavefunction for a particle with \par
\noindent mass $m$.
So $\Delta _p^{\frac{1}{3}} \sum_l \frac{\delta}{\delta q_l}
\left(\Delta_p^{-\frac{1}{2}}\sum_k\alpha_{lk} \delta\psi/\delta
q_k\right) = \frac{1}{m}\nabla^2 \psi$. The second term is an
arbitrary potential, and
the third term is the time-derivative. Note that since Schr\"odinger
only attributes meaning to the square modulus of the wavefunction,
the sign of this last term is arbitrary. He chooses it to be positive.
This can be rewritten:

\begin{equation}
  -\frac{\hbar^2}{2m}\nabla^2 \psi
  + V(\vec r, t)\psi  = i\hbar\frac{\delta \psi}{\delta t}
\end{equation}

Introducing the Hamiltonian $\hat H$:

\begin{equation}
  \hat H\psi  = i\hbar\frac{\delta \psi}{\delta t}
  \label{eq:tdse}
\end{equation}

\subsection{Solving the Schr\"odinger equation for an atomic system}

The following explanation is heavily inspired by Sholl \cite{sholl}.
The Schro\"odinger equation is mathematically a well-behaved partial
differential equation. Most importantly, it is linear. This allows it to
be solved numerically with relative ease, theoretically. However, we quickly
run into a problem when we attempt to work in a many-body problem like
a molecular system. Consider the Hamiltonian in
\Cref{eq:tise,eq:tdse}, but rewritten for a system containing $N$ electrons
and $M$ nuclei:

\begin{equation}
  \hat H =
  \underbrace{
    -\frac{\hbar^2}{2m}\sum_{i=1}^N\nabla_i^2
  }_{\text{Electron kinetic energy}} -
  \underbrace{
    \frac{\hbar^2}{2}\sum_{A = 1}^M\frac{1}{M_A}\nabla_A^2
  }_{\text{Nuclei kinetic energy}} -
  \underbrace{\sum_{i=1}^N\sum_{A=1}^M \frac{e^2Z_A}{4\pi \varepsilon_0r_{iA}}
  }_{\text{Electron-nuclei attraction}} +
  \underbrace{\sum_{i = 1}^N \sum_{j > 1}^N \frac{e^2}{4\pi \varepsilon_0r_{ij}}
  }_{\text{Electron-electron repulsion}} +
  \underbrace{
    \sum_{A = 1}^M \sum_{B>A}^M \frac{e^2Z_AZ_B}{4\pi \varepsilon_0R_{AB}}
  }_{\text{Nuclei-nuclei repulsion}}
\end{equation}

where $M_A$ is the mass of a nuclei, $Z_A, Z_B$ is the atomic number
of a nuclei,
$r_{iA}$ is the distance of an electron from a nuclei, $r_{ij}$ is the distance
of an electron from another electron, $\varepsilon_0$ is the vacuum
permittivity and $R_{AB}$ is the distance
between nuclei. This is a many-body problem, and can easily grow to become
impossible to compute numerically. For example, each particle can be represented
with three spacial coordinates, and one time coordinate. If we split the space
we are solving for into a 10x10x10 grid with 10 time-steps, assuming a simple
single-atom system with one nuclei and 20 electrons, we find that this is an
84-dimensional ($21\times4$) problem. This results in having to
calculate $(10^4)^{84}$
data points, or \num{e336}. This number is significantly larger than the number
of particles in the universe ($\sim\num{e80}$) \cite{eddington}.

\subsubsection{Born-Oppenheimer approxmiation}

One of the earliest approximations used to reduce the computational complexity
of this problem was posed by Max Born and Robert J. Oppenheimer in
1927 \cite{born-oppenheimer}. This approximation in essence claims that
the electronic wavefunctions and the nuclei wave functions can be separated and
solved separately due to the large mass difference between electrons and
nuclei. This allows the electronic wavefunctions to be calculated by
assuming that nuclei are fixed in place. There are arguments going on
even in recent years about the quantum-mechanical validity of this approach,
but it is nonetheless widely accepted\cite{huggett}. \par

From a practical point of view, this allows nuclei to be "fixed" in place, while
the electronic wavefunctions are solved separately. Then once the electronic
wavefunctions are solved, the nuclear wavefunctions are solved. This
is repeated for small time steps, enabling time evolution without having to
simultaneously solve for the electronic and nuclear wavefunctions. The
\textit{electronic Hamiltonian} can be rewritten:

\begin{align}
  \hat H_{\text{elec}} &=
  \overbrace{\underbrace{
      -\frac{\hbar^2}{2m}\sum_{i=1}^N\nabla_i^2
  }_{\text{Electron kinetic energy}}}^{\hat T}
  \overbrace{- \underbrace{0}_{\text{No nuclear kinetic energy}} -
    \underbrace{\sum_{i=1}^N\sum_{A=1}^M \frac{e^2Z_A}{4\pi \varepsilon_0r_{iA}}
  }_{\text{Electron-nuclei attraction}}}^{\hat V_{Ne}} +
  \overbrace{\underbrace{\sum_{i = 1}^N \sum_{j > 1}^N
      \frac{e^2}{4\pi \varepsilon_0r_{ij}}
    }_{\text{Electron-electron repulsion}} +
  \underbrace{C}_{\text{Constant nuclear repulsion}}}^{\hat V_{ee}} \\
  \hat H_{\text{elec}} &= \hat T + \hat V_{Ne} + \hat V_{ee}
  \label{eq:electronic hamiltonian}
\end{align}

In practice, this allows one to solve for the \textit{electronic
wavefunction}.
The Schr\"odinger equation with
the electronic Hamiltonian can be written (For simplicity, we ignore
spin, $\vec x_i = \vec r_i s_i$).):

\begin{equation}
\hat H_{\text{elec}}\psi_{\text{elec}}(\vec r_1,\dots,\vec r_N) =
E_{\text{elec}}\psi_{\text{elec}}(\vec r_1, \dots, \vec r_N)
\end{equation}

Often the objective of a DFT calculation is to calculate the
\textit{ground state} electronic
wavefunction. This is the electronic wavefuction that yields the lowest
energy. Unfortunately,this approximation is not enough to make the
solution calculable. The $\hat V_{ee}$ term in \Cref{eq:electronic hamiltonian}
still makes this a many-body problem. Another approxmiation is needed.

\subsubsection{Kohn-Sham equations}
\label{sec:kohn-sham}
In theory, the probability that a number of electrons, $N$, at any
one of a set of coordinates,
$\vec r_1,\dots,\vec r_N$, can be measured, while the actual
wavefunction cannot. This probability can be written:

\begin{equation}
|\psi(\vec r_1, \dots, \vec r_N)|^2 = \psi^*(\vec r_1, \dots, \vec
r_N)\psi(\vec r_1,\dots, \vec r_N) \\
\end{equation}

One way of simplifying this problem is by employing a technique developed
by Douglas Hartree\cite{hartree}. This involves expressing the electronic
wavefunction as a product of individual electron wavefunctions.

\begin{equation}
\psi(\vec r_1, \dots, \vec r_N) \cong  \psi_1(\vec r_1)\psi_2(\vec
r_2)\dots\psi_N(\vec r_N)
\end{equation}

This approxmiation can be explored and expanded to eventually reach the
Hatree-Fock method, another approxmiation method that is widely used for
solving the Schr\"odinger equation. However, it is separate from DFT. A quantity
that is analogous to the Hatree product is the electron density:

\begin{equation}
n(\vec r) = 2 \sum_i\psi_i^*(\vec r)\psi_i(\vec r)
\label{eq:density}
\end{equation}

This eliminates the uniqueness of each electron, but in exchange greatly reduces
the dimensionality of the $n(\vec r)$ quantity compared to the
wavefunction ($3N \rightarrow 3$). The
factor of 2 is a consequence of the fact that each eigenstate can be occupied
by two electrons, one spin-up, another spin-down. The core of DFT is
to solve not
for the electronic wavefunction, but instead solve for the electron
density. This
is enabled by two mathematical theorems, proved  by Walter Kohn and
Pierre Hohenberg \cite{hohenberg_kohn}.
The first theorem is: \textit{The ground-state energy from Schr\"odinger's
equation is a unique functional of the electron density.} This allows the
electron density to be used to solve for ground-state energy.

One may ask, what is a \textit{functional}? A functional is a mathematical
operation that takes a function and outputs a number. For example:

\begin{equation}
F[f] = \int_{-1}^1 f(x)dx
\end{equation}

\noindent is a functional. If you plug in $f(x) = x^2$:

\begin{align*}
F[x^2] &= \int_{-1}^1 x^2 dx \\
F[x^2] &= \frac{x^3}{3}\Biggr|_{-1}^1 \\
F[x^2] &= \frac{2}{3} \\
\end{align*}

\noindent a number comes out as a result.

\noindent To determine the form of the electron functional, a second
theorem tells
us: \textit{the electron density that minimizes the energy of
the overall functional is the true electron density corresponding to
the full solution of the Schrödinger equation.} Assuming the functional
is known, this allows one to vary the density to find the minimal energy,
which corresponds to the ground-state energy, as per the first theorem.

A derivation of equations is provided by Walter Kohn and Lu Jeu Sham
\cite{kohn_sham}. They found that the energy functional can be written as:

\begin{equation}
E[\{\psi_i\}] = E_{\text{known}}[\{\psi_i\}] + E_{\text{XC}}[\{\psi_i\}]
\end{equation}

\noindent $E_{\text{known}}$ function can be written:

\begin{equation}
E_{\text{known}}[\{\psi_i\}] =
\underbrace{\frac{h^2}{m}\sum_i\int \psi_i^*\nabla^2\psi_id^3
r}_{\text{Electron kinetic}} +
\underbrace{\int V(r)n(r)d^3r}_{\text{Electron-nuclei interaction}} +
\underbrace{\frac{e^2}{2}\int\int\frac{n(r)n(r')}{|r - r'|}d^3r d^3
r}_{\text{Electron-electron interaction}} +
\underbrace{E_{\text{ion}}}_{\text{Nuclei-nuclei interaction}}
\label{eq:known}
\end{equation}

\noindent The other functional is $E_{\text{EX}}[\psi_i]$. This is the
\textit{exchange-correlation functional}, which accounts for all other
quantum-mechanical effects not found in $E_{\text{known}}[\psi_i]$. Presume for
now that this functional is known. All that is left is to determine the form
for the electron density. Fortunately, the Kohn--Sham equations provide a method
to do this:

\begin{equation}
\left[ \frac{h^2}{2m}\nabla^2 + V(\vec r) + V_H(\vec r) + V_{XC}(\vec r)
\right]\psi_i(\vec r) = \varepsilon_i\psi_i(\vec r)
\label{eq:kohn-sham}
\end{equation}

$V(r)$ is the same as in \Cref{eq:known}, while $V_H(r)$ is known as the Hartree
potential:

\begin{equation}
V_H(\vec r) = e^2 \int\frac{n(r')}{|r - r'|} d^3r
\end{equation}

\noindent which accounts for repulsion by the surrounding electron density. It
also accounts for self-interaction, which is not physically accurate and must be
corrected for in $V_{XC}$. This can be written as the "functional derivative" of
the exchange-correlation functional:

\begin{equation}
V_{XC}(r) = \frac{\delta E_{XC}(r)}{\delta n(r)}
\label{eq:Vxc}
\end{equation}

\subsubsection{Procedure for DFT}

Using the Kohn-Sham equations we can now establish a general procedure for
performing a DFT calculations. The following procedure can be followed:

\begin{enumerate}
\item Define a trial electron density, $n(r)$.
\item Solve the Kohn-Sham equations using the trial density
  (\Cref{eq:kohn-sham}). \label{en:solve}
\item Calculate the electron density using \Cref{eq:density}.
\item Compare the calculated density with the original density. If they match,
  the ground state as been found. Otherwise, "update" the electron
  density and repeat step
  \ref{en:solve}.
\end{enumerate}

\noindent This general procedure omits several details, such as how
to choose the
trial density, what criteria are used to compare the densities, and
how to "update"
the electron densities. These details are not expanded upon here, but
can be found
in more detailed books on DFT.

\subsubsection{The exchange-correction functional}
\label{sec:Exc}

A major detail that was previously omitted is what the true form of the
exchange-correlation functional is. This is a central problem in DFT,
and its true form is  still not fully understood. There are a number of
approximations for the exchange-correlation functional that have
emerged over the years. A simple case where this functional as been derived
is the exchange-correlation functional for a uniform electron gas.
We set the exchange correlation covered
in \Cref{eq:Vxc} to a known exchange-correlation potential corresponding to
an electron density at a particular position.
\begin{equation}
V_{\text{XC}}(r) = V_{\text{XC}}^{\text{electron gas}}[n(r)]
\end{equation}

The local density is used to
approximate the exchange-correlation function. Hence, it is called the
\textit{local density approximation} (LDA). While this provides an approximation
for a solution to the Schr\"odinger equation, it is not exact due to the
LDA. Another approximation which also accounts for the local gradient of the
electron density is the \textit{generalized gradient approximation}. While this
utilizes more information than the LDA, it is not always more accurate. Many
other functionals have been found that work well in various domains. There have
been many forms of the GGA, such as the Perdew-Wang 91 functional (PW91), which
works well in solids \cite{perdew-wang}. Many other GGA functionals have been
created for various use cases, such as in isolated molecules.

\subsection{Downsides of DFT}

Because DFT is inherently an approximation and not an exact solution to the
Schr\"odinger equation, it has some associated error. This error cannot always
be quantified, and there are specific areas where DFT falls short \cite{sholl}.
For example, DFT is not reliable for calculating electronic exited states. This
is because the Hohenberg-Kohn theorems mentioned in \Cref{sec:kohn-sham} only
apply to the ground electronic states, and not excited states. In addition, it
tends to underestimate band gaps in insulators and semi-conductors.
DFT also has challenges modeling weak van der Waals interactions that result
from long-distance electron correlation. Recall from \Cref{sec:Exc} that
estimates for the exchange-correlation functional generally use local
approximations, making them poorly suited for long-range effects. Finally,
DFT is limited to relatively small collections of atoms. While DFT enables
the simulation of many more atoms than direct solutions of the Schr\"odinger
equation (which quickly becomes infeasible even for single small atoms), even
a recent 2023 study pushing the limits of DFT was only able to simulate 100,000
atoms \cite{dogan}. While this is a significance achievement, it is still far
from macroscopic levels. Utilizing DFT requires techniques for extracting
macroscopic properties from simulations of relatively  small
collections of atoms.

\section{A working example}

Now that the basics of DFT have been established, this section will cover a
simple working example to demonstrate how to perform a DFT calculation.
Naturally, this requires picking a specific computational framework to
perform the operation.
\href{https://www.quantum-espresso.org/}{Quantum Espresso} will be used in
this work, as it is a mature, well-trusted, featureful, and open-source
program that is accessible to anyone with a modern computer. The working
example will be kept small to enable even modest hardware to execute it.

\begin{figure}
\centering
% set view angle
\tdplotsetmaincoords{75}{125}% rot x rot z

\begin{tikzpicture}[tdplot_main_coords,scale=3.0, axis/.style={thick,
  ->, >=stealth'}]

  \def \s {0.0}     % defines shear of z along y (instead of angles),
  % for orthogonal systems: 0
  \def \su {-0.0}   % defines shear of x along y (instead of angles),
  % for orthogonal systems: 0
  \def \a {1}       % unit cells along a (only integers!)
  \def \b {1}       % unit cells along b (only integers!)
  \def \c {1}       % unit cells along c (only integers!)

  % lattice directions parallel to c
  \foreach \u in {0,1,...,\a}
  \foreach \v in {0,1,...,\b}
  \foreach \w in {0,1,...,\c}
  \draw[very thin,gray] (\u+\su*\v,\v,0) -- (\u+\su*\v,\v+\s*\w,\w);
  % lattice directions parallel to b
  \foreach \u in {0,1,...,\a}
  \foreach \v in {0,1,...,\b}
  \foreach \w in {0,1,...,\c}
  \draw[very thin,gray] (\u,0+\s*\w,\w) -- (\u+\su*\v,\v+\s*\w,\w);
  % lattice directions parallel to a
  \foreach \u in {0,1,...,\a}
  \foreach \v in {0,1,...,\b}
  \foreach \w in {0,1,...,\c}
  \draw[very thin, gray] (0+\su*\v,\v+\s*\w,\w) -- (\u+\su*\v,\v+\s*\w,\w);

  % basis vectors
  \draw[axis] (0,0,0) -- (1,0,0) node[left] {$\bf x$};
  \draw[axis] (0,0,0) -- (\su,1,0) node[right] {$\bf y$};
  \draw[axis] (0,0,0) -- (0,\s,1) node[above] {$\bf z$};
  \foreach \u in {0,1,...,\a}
  \foreach \v in {0,1,...,\b}
  \foreach \w in {0,1,...,\c}
  \draw plot [mark=*, mark size=.5] coordinates{(\u+\su*\v,\v+\s*\w,\w)};

  % lattice vectors
  \draw[thick,red] (0,0,0) -- (0+\su*3,1+\s*1,1) node[right] {\hkl[011]};
  \draw[thick,red] (0,0,0) -- (1+\su*3,1+\s*1,0) node[right] {\hkl[110]};
  \draw[thick,red] (0,0,0) -- (1+\su*3,0+\s*1,1) node[right] {\hkl[101]};

\end{tikzpicture}
\caption{Diagram of a simple cubic structure with lattice vectors
\hkl[011], \hkl[101], and \hkl[110] shown.}
\label{fig:lattice}
\end{figure}

\subsection{Lattice structure}

This example will model a simple cubic lattice structure. The structure is
shown in \Cref{fig:lattice}. Note that this structure is very rare in nature.
The only element that exhibits this structure is polonium \cite{beamer}.
However, this structure is convenient as the positions of each atom in the
unit cell can be represented by lattice vectors with coordintes being
combinations
of 0 and 1.

\printbibliography

\end{document}
